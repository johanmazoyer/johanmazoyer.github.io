% LaTeX file for resume
\documentclass[11pt, a4paper, french]{article}
\usepackage[main=french, provide=*]{babel}
\usepackage[utf8]{inputenc}


\usepackage[usenames, dvipsnames]{xcolor}

\usepackage[framemethod=tikz]{mdframed}
\newmdenv[innerlinewidth=0.8pt, roundcorner=4pt,linecolor=RoyalBlue,innerleftmargin=6pt,
innerrightmargin=6pt,innertopmargin=6pt,innerbottommargin=6pt]{mybox}
\usepackage{wrapfig}


\usepackage{etaremune}


\usepackage[colorlinks = true,urlcolor = BrickRed]{hyperref}

\usepackage{eurosym}


\usepackage{datetime}
\newdateformat{yeardate}{\THEYEAR}

\usepackage{fancyhdr}  % use this package to get a 2 line header
\renewcommand{\headrulewidth}{0pt} % suppress line drawn by default by fancyhdr

\pagestyle{fancy}     % set pagestyle for document
\usepackage{lastpage}
% \cfoot{ \thepage}
% \setcounter{page}{3}
\rhead{\textcolor{gray}{ \thepage/\pageref*{LastPage}}}
\cfoot{}
\rfoot{\href{https://johanmazoyer.com/}{johanmazoyer.com}}
\rhead{}


\setlength{\headwidth}{17.2cm} % taille de l'entete
\usepackage[total={17.2cm,25.cm}, left=1.9cm, top=2.5cm]{geometry}
\setlength{\parindent}{0pt} 


\begin{document}
\lfoot{\textcolor{Gray}{CV mis à jour en \yeardate\today}}

\begin{huge}
\noindent\textbf{Johan MAZOYER}
\end{huge}\\

\textbf{Intérêts de recherche:} Instrumentation Optique, Imagerie Directe et Coronographie,
Observation et Caractérisation de Systèmes Extrasolaires, Disques de Débris\\



%%%%%%%%%%%%%%%%%%%%%%%%%%%%%%%%%%%%%%%%%%%%%%%%%%%%%%%%%%%%%%%%%%%%%%%%%%%%%%%%%%%%%%%%%%%%%%%%%%%%%%%%%%%%%%%%%%%%%%%%
%%%%%% SECTION RESEARCH EXPERIENCE
%%%%%%%%%%%%%%%%%%%%%%%%%%%%%%%%%%%%%%%%%%%%%%%%%%%%%%%%%%%%%%%%%%%%%%%%%%%%%%%%%%%%%%%%%%%%%%%%%%%%%%%%%%%%%%%%%%%%%%%%


\vspace{-1.1cm}
\textcolor{RoyalBlue}{\section{\large EXPÉRIENCES PROFESSIONNELLES}
\vspace{-0.2cm}\hrule}
\vspace{0.4cm}


\textbf{Chargé de recherche CNRS} --
\href{https://LIRA.obspm.fr/}{\textbf{LIRA/Observatoire de Paris - PSL}} (France)
\hfill     	 { \bf Depuis 2020}\\

\vspace{-0.11cm}
\textbf{Carl Sagan Fellow} --
\href{https://www.jpl.nasa.gov/}{\textbf{NASA Jet Propulsion Laboratory}} (Pasadena, CA)
\hfill      { \bf 2018-2019}\\

\vspace{-0.11cm}
\textbf{Post-doctorant} --
\href{http://physics-astronomy.jhu.edu/}{\textbf{Johns Hopkins University}} (Baltimore, MD)
\hfill   	 { \bf 2016-2018}\\

\vspace{-0.11cm}
\textbf{Post-doctorant} --
\href{http://www.stsci.edu}{\textbf{Space Telescope Science Institute}} (Baltimore, MD)
\hfill        { \bf 2014-2016}\\\

\vspace{-0.11cm}
\textbf{Doctorant} --
\href{https://LIRA.obspm.fr/}{\textbf{LIRA/Observatoire de Paris - PSL}} (France)
\hfill        { \bf 2011-2014}\\


% \textbf{Jet Propulsion Laboratory} (JPL)  \hfill       {\small Pasadena, CA, USA} \\
% {\small \textit{Carl Sagan Fellow}}  \hfill  	 {\small \bf 2018 -- Présent}\\

% \vspace{-0.2cm}
% \hspace{0.5cm} \parbox{0.9\linewidth}{
%    \begin{itemize} \itemsep -3pt % reduce space between items
%    \vspace{-0.4cm}
%     \item \small Astrophysique : Caractérisation de disques avec GPI et WFIRST
%    \item \small Instrumentation : Développement de techniques instrumentales sur banc optique
%  \end{itemize} }\\

% \textbf{Johns Hopkins University} (JHU) \hfill       {\small Baltimore, MD, USA} \\
% {\small \textit{Chercheur post-doctoral}; superviseur : Christine Chen}  \hfill  	 {\small \bf 2016 -- 2018}\\

% \vspace{-0.4cm}
% \hspace{0.5cm} \parbox{0.9\linewidth}{
%    \begin{itemize} \itemsep -3pt % reduce space between items
%    \vspace{-0.2cm}
%     \item \small Astrophysique : Large programme sur les disques de débris sur GPI
%    \item \small Instrumentation : Méthodes deux miroirs pour l'imagerie haut-contraste
%  \end{itemize} }\\

% \vspace{-0.2cm}
% \textbf{Space Telescope Science Institute} (STScI) \hfill       {\small Baltimore, MD, USA} \\
% {\small \textit{Chercheur post-doctoral}; superviseur : Laurent Pueyo}  \hfill  	 {\small \bf 2014 --2016}\\

% \vspace{-0.4cm}
% \hspace{0.5cm} \parbox{0.9\linewidth}{
%    \begin{itemize} \itemsep -3pt % reduce space between items
%    \vspace{-0.2cm}
%     \item \small Astrophysique : Imagerie directe de disques (NICMOS, NICI, SPHERE)
%    \item \small Instrumentation : Correction active d'ouvertures de télescopes complexes \\

%  \end{itemize} }

% \vspace{-0.2cm}
% \textbf{Laboratoire d'études spatiales et d'instrumentation en astrophysique} (LIRA)  \hfill        {\small Paris, FR} \\
% {\small \textit{Doctorant}; encadrants : Pierre Baudoz et Gérard Rousset}  \hfill  	 	 {\small \bf 2011 -- 2014}\\
% \vspace{-0.1cm}
% \hspace{0.5cm} \parbox{0.9\linewidth}{
% \vspace{0.2cm}
% \small  Développement d'un senseur de front d'onde haut-contraste et imagerie\\ de disques de débris (Gemini/NICI)} \\

% \vspace{0.2cm}
% \textbf{Los Alamos National Laboratory} (LANL) \hfill    	Los Alamos, NM, USA\\
% {\small \textit{Étudiant de M2}; encadrants : Roger C. Wiens et Jérémie Lasue
%  \hfill  		 {\bf 2011}}\\
%  \vspace{-0.4cm}
% \hspace{0.5cm} \parbox{0.9\linewidth}{
% \small  MSL/ChemCam : Influence de l'atmosphère martienne sur les limites de détection} \\

% \vspace{0.1cm}
% \textbf{Institut de Recherche en Astrophysique et Planétologie} (IRAP) \hfill    	Toulouse, FR\\
% {\small \textit{Étudiant de M2}; encadrants:  Sylvestre Maurice et Olivier Gasnault
%  \hfill  		 {\bf 2011}}\\
%  \vspace{-0.4cm}
% \hspace{0.5cm} \parbox{0.9\linewidth}{
% \small  MSL/ChemCam : Influence de l'atmosphère sur les mesures du LIBS} \\

%\vspace{0.1cm}
% \textbf{CNES} \hfill    	Toulouse, France\\
%{\small \textit{Étudiant de M1}; Satellites Pleiades : simulation de composants de vol\hfill  		 {\bf Mars -- Juil. 2010}}\\
%
%\vspace{-0.3cm}
%\textbf{Le Relais} (Emmaüs) \hfill    		  Koudougou, Burkina Faso\\
%{\small Stage humanitaire\hfill  		  {\bf Juil. -- Sept. 2009}}	\\

%%%%%%%%%%%%%%%%%%%%%%%%%%%%%%%%%%%%%%%%%%%%%%%%%%%%%%%%%%%%%%%%%%%%%%%%%%%%%%%%%%%%%%%%%%%%%%%%%%%%%%%%%%%%%%%%%%%%%%%%
%%%%%% SECTION EDUCATION
%%%%%%%%%%%%%%%%%%%%%%%%%%%%%%%%%%%%%%%%%%%%%%%%%%%%%%%%%%%%%%%%%%%%%%%%%%%%%%%%%%%%%%%%%%%%%%%%%%%%%%%%%%%%%%%%%%%%%%%%

\vspace{-0.43cm}
\textcolor{RoyalBlue}{\section{\large FORMATION}
\vspace{-0.2cm}\hrule}
\vspace{0.4cm}


{\bf HDR} -- \href{https://observatoiredeparis.psl.eu/}{\textbf{Observatoire de Paris - PSL}}
\hfill  { \bf Mars 2024}\\

\vspace{-0.11cm}
{\bf Doctorat} -- Astronomie et Astrophysique -- \href{https://u-paris.fr/}{\textbf{Université Paris Cité}} \hfill  {  \bf  Sept. 2014}\\
{\footnotesize
\it Thèse: Haut contraste pour l'imagerie directe d'exoplanètes et de disques (P. Baudoz \& G. Rousset)}\\

\vspace{-0.11cm}
{\bf Master} -- Astrophysique et planétologie --
\href{https://www5.obs-mip.fr/masterasep/}{\textbf{Université de Toulouse}}  \hfill  { \bf Sept. 2011}\\
{\footnotesize
\it Thèse: Influence de l'atmosphère martienne sur les perf. de MSL/Chemcam (O. Gasnault \& R. Wiens)} \hfill { \small \bf }\\

\vspace{-0.11cm}
{\bf Diplôme d'ingénieur} -- Techniques d'Imagerie Spatiale -- \href{https://www.isae-supaero.fr/isae-supaero-6/}{\textbf{\textbf{ISAE Supaero}}} \hfill { \bf Sept. 2011}\\


\vspace{-0.11cm}
{\bf Diplôme d'ingénieur} -- Systèmes Embarqués --
\href{https://www.polytechnique.edu/}{\textbf{\textbf{Ecole polytechnique}}}
\hfill \hfill { \bf Sept. 2011}  \\

%%%%%%%%%%%%%%%%%%%%%%%%%%%%%%%%%%%%%%%%%%%%%%%%%%%%%%%%%%%%%%%%%%%%%%%%%%%%%%%%%%%%%%%%%%%%%%%%%%%%%%%%%%%%%%%%%%%%%%%%
%%%%%% SECTION AWARDS
%%%%%%%%%%%%%%%%%%%%%%%%%%%%%%%%%%%%%%%%%%%%%%%%%%%%%%%%%%%%%%%%%%%%%%%%%%%%%%%%%%%%%%%%%%%%%%%%%%%%%%%%%%%%%%%%%%%%%%%%

\vspace{-0.43cm}
\textcolor{RoyalBlue}{\section{\large BOURSES \& PRIX}
\vspace{-0.2cm}\hrule}

\vspace{0.4cm}
\textbf{ERC - Consolidator Grant (PI)} ECHOES - 2 M€ \hfill \textbf{Depuis 2026}\\ 

\vspace{-0.2cm}
\textbf{ANR JCJC (PI)} - 370 k€ (interrompue et remboursée) \hfill \textbf{2025-2026}\\ 

\vspace{-0.2cm}
\textbf{DIM Origins (PI)} Fonds pour l'achat de matériel (spatial light modulator) - 20 k€ \hfill \textbf{2023}\\ 

\vspace{-0.2cm}
\textbf{CNES (co-PI)} Bourse CNES d'Iva Laginja - 60 k€/an \hfill \textbf{2022}\\ 

\vspace{-0.2cm}
\textbf{Data Intensive Artificial Intelligence (PI)} Bourse thèse de Y. Gutierrez - 120 k€ / 3 ans \hfill \textbf{2021}\\ 

\vspace{-0.2cm}
\textbf{Programme \href{https://www.univ-paris13.fr/ecos-sud/}{EcosSud} (PI)} collaboration France-Chili avec {\it Universidad de Chile} -- 50 k€ \hfill   \textbf{2020}\\ %- \$300K/3 yrs

\vspace{-0.2cm}
\textbf{NASA Group Award}: LBTI Hosts Survey Science Team \hfill   \textbf{2020}\\ 

\vspace{-0.2cm}
\textbf{Carl Sagan Fellowship (PI)} (\href{http://www.stsci.edu/stsci-research/fellowships/nasa-hubble-fellowship-program}{NASA Hubble Fellowship Program}) -- 280k€/3 ans \hfill   { \bf 2018}\\

\vspace{-0.2cm}
Couverture du journal \textbf{Astronomy \& Astrophysics} (\href{https://www.aanda.org/articles/aa/abs/2014/04/contents/contents.html}{Volume 564}) \hfill  { \bf 2014}\\

\vspace{-0.15cm}
\textbf{Prix meilleur poster}, conférence des chercheurs du CNES (JC2) \hfill   { \bf 2013}\\

\vspace{-0.2cm}
\textbf{Bourse doctorale (PI)} du CNES -- 120 k€/3ans \hfill   { \bf 2011}\\

% \vspace{-0.15cm}
% \textbf{Bourse d'étude} de l'Ecole polytechnique -- 4 ans \hfill   { \bf 2007}\\

%%%%%%%%%%%%%%%%%%%%%%%%%%%%%%%%%%%%%%%%%%%%%%%%%%%%%%%%%%%%%%%%%%%%%%%%%%%%%%%%%%%%%%%%%%%%%%%%%%%%%%%%%%%%%%%%%%%%%%%%
%%%%%% SECTION VULGARISATION
%%%%%%%%%%%%%%%%%%%%%%%%%%%%%%%%%%%%%%%%%%%%%%%%%%%%%%%%%%%%%%%%%%%%%%%%%%%%%%%%%%%%%%%%%%%%%%%%%%%%%%%%%%%%%%%%%%%%%%%%

\newpage
\textcolor{white}{.}
\vspace{-1.5cm}
\textcolor{RoyalBlue}{\section{\large DIFFUSION DES SCIENCES}
\vspace{-0.2cm}\hrule}
\vspace{0.3cm}
\begin{wrapfigure}{r}{0.16\textwidth}
\vspace{-0.8cm}
\begin{mybox}
\includegraphics[width=1.\textwidth]{figures_CV/PodcastScience.png}
\end{mybox}
\vspace{-0.9cm}
\end{wrapfigure}


Je suis très impliqué dans la vulgarisation scintifique francophone. En plus d'interventions fréquentes en classe ou grand public, 
j'organise régulièrement des événements institutionnels ou associatifs:\\ 
$\bullet$ \textbf{Podcast Science}: J'anime chaque semaine \href{http://www.podcastscience.fm}{\textbf{PodcastScience.fm}},
un programme scientifique généraliste diffusé chaque semaine.
Écouté par 10 à 20 000 auditeurs, il a reçu le Golden Blog Award du meilleur blog scientifique en 2012.\\
$\bullet$ \textbf{Les p'tits cueilleurs d'étoiles}: Association organisant des visites d'astronomes dans les hôpitaux pour 
enfants. J'organise les visites dans la région parisienne (25 visites/an).\\
$\bullet$ \textbf{Fête de la science}: J'ai été l'organisateur principal des journées portes ouvertes annuelles 
de l'Observatoire de Paris ($\sim$1000 visiteurs/an) pendant deux années consécutives (2023 et 2024).\\

%%%%%%%%%%%%%%%%%%%%%%%%%%%%%%%%%%%%%%%%%%%%%%%%%%%%%%%%%%%%%%%%%%%%%%%%%%%%%%%%%%%%%%%%%%%%%%%%%%%%%%%%%%%%%%%%%%%%%%%%
%%%%%% campagnes d’observations
%%%%%%%%%%%%%%%%%%%%%%%%%%%%%%%%%%%%%%%%%%%%%%%%%%%%%%%%%%%%%%%%%%%%%%%%%%%%%%%%%%%%%%%%%%%%%%%%%%%%%%%%%%%%%%%%%%%%%%%%

% \vspace{-0.4cm}
% \textcolor{RoyalBlue}{\section{\large CAMPAGNES D'OBSERVATIONS}
% \vspace{-0.3cm}\hrule}
% \vspace{0.3cm}

% \textbf{Palomar Observatory (200 inch telescope)}
% \begin{itemize} \itemsep -1pt % reduce space between items
% 	    \item \small Mai 2015, 3 nuits -- Premiers tests sur télescope de la Self-Coherent Camera
% \end{itemize}

% \textbf{Gemini South/GPI}
% 	\begin{itemize} \itemsep -1pt % reduce space between items
% 	    \item \small Décembre 2015 - Novembre 2018 --	25 nuits en cumulé pour le large programme disque ou le programme de temps garanti du consortium GPI. Depuis fin 2016, toutes les observations Gemini sud sont menées à distance. J'ai effectué la majorité de mes observations à distance depuis Berkeley, le JPL et le STScI.
% \end{itemize}



%%%%%%%%%%%%%%%%%%%%%%%%%%%%%%%%%%%%%%%%%%%%%%%%%%%%%%%%%%%%%%%%%%%%%%%%%%%%%%%%%%%%%%%%%%%%%%%%%%%%%%%%%%%%%%%%%%%%%%%%
%%%%%% Demandes de temps
%%%%%%%%%%%%%%%%%%%%%%%%%%%%%%%%%%%%%%%%%%%%%%%%%%%%%%%%%%%%%%%%%%%%%%%%%%%%%%%%%%%%%%%%%%%%%%%%%%%%%%%%%%%%%%%%%%%%%%%%

% \newpage
% \textcolor{white}{.}
% \vspace{-0.8cm}
% \textcolor{RoyalBlue}{\section{\large DEMANDES DE TEMPS D'OBSERVATION ACCEPTÉES}
% \vspace{-0.3cm}\hrule}
% \vspace{0.3cm}

% \textbf{GEMINI South/GPI} : Membre du consortium GPI depuis septembre 2016
% \vspace{0.3cm}
% \begin{itemize} \itemsep -1pt % reduce space between items
%     \item \small GS-2015B-LP-6 ``Characterizing Dusty Debris in Exoplanetary Systems'' (PI: C. Chen)
%     \item \small DT-2019A-009 ``Decoding the Asymmetric Scattered Light Around HD 15115'' (PI: C. Chen)
%     \item \small GS-2019A-Q-109 ``Completing a Survey for Resolved Debris Disks in the Sco-Cen Assoc.'' (PI: J. Patience)
% \end{itemize}


% \textbf{VLT/SPHERE}
% %\vspace{-0.2cm}
% \begin{itemize} \itemsep -1pt % reduce space between items
% 	\item \small P 0101.C-0128 ``Resolving multiple belts and sub-structures in inner regions of highly inclined debris disks" (PI : A. Boccaletti)
%     \item \small P 098.C-0686 ``Resolving sub-structures in rings and gaps of inclined debris disks" (PI : A. Boccaletti)
%     \item \small P 096.C-0640 ``Exploring the inner cavities of two very inclined debris disks'' (PI : A. Boccaletti)
%     \item \small P 095.C-0381 ``Investigating the inner part of a transitional disk" (PI : A. Boccaletti)
% \end{itemize}

% \textbf{JWST / MIRI, NIRCam, NIRSPEC \& NIRISS}
% %\vspace{-0.2cm}
% \begin{itemize} \itemsep -1pt % reduce space between items
%     \item \small Programme Early realease Science (ERS) ``High Contrast Imaging of Exoplanets and Exoplanetary Systems with JWST" (PI : Sasha Hinkley)
% \end{itemize}

%%%%%%%%%%%%%%%%%%%%%%%%%%%%%%%%%%%%%%%%%%%%%%%%%%%%%%%%%%%%%%%%%%%%%%%%%%%%%%%%%%%%%%%%%%%%%%%%%%%%%%%%%%%%%%%%%%%%%%%%
%%%%%% SECTION RESPONSABILITE
%%%%%%%%%%%%%%%%%%%%%%%%%%%%%%%%%%%%%%%%%%%%%%%%%%%%%%%%%%%%%%%%%%%%%%%%%%%%%%%%%%%%%%%%%%%%%%%%%%%%%%%%%%%%%%%%%%%%%%%%
\lfoot{}


\vspace{-0.5cm}
\textcolor{RoyalBlue}{\section{\large ACTIVITÉS POUR LA COMMUNAUTÉ}
\vspace{-0.2cm}\hrule}
\vspace{0.4cm}
% \lhead{\textcolor{gray}{J. MAZOYER}}

\textbf{Responsibilities in scientific instruments:}
\begin{itemize} \itemsep -2pt
    \item[$\bullet$] \textbf{Roman Space Telescope Coronagraph}: Représentant adjoint du CNES \hfill Depuis 2023
    \item[$\bullet$] \textbf{VLT/SPHERE\texttt{+}}: Responsable du groupe de travail \textit{Dark-Hole} \hfill Depuis 2022
    \item[$\bullet$] \textbf{Habitable Exoplanet Observatory (HabEx)}: Contributeur scientifique  \hfill  2019
    \item[$\bullet$] \textbf{Large UV Optical Infrared Surveyor (LUVOIR)}: Contributeur scientifique \hfill 2019
    \item[$\bullet$] \textbf{Gemini Planet Imager (GPI)}: Membre junior du consortium \hfill 2017-2020
\end{itemize}

\vspace{0.1cm}
\textbf{Organisation de conférences, ateliers}
\vspace{-0.1cm}
\begin{itemize} \itemsep -2pt
    \item[$\bullet$] Roman coronagraphic instrument summer school (SOC) \hfill  Nice, 2026
    \item[$\bullet$] ExoSystèmes 4 (SOC) \hfill Lyon, 2024 
    \item[$\bullet$] National Capital Area Disks conference (SOC \& LOC) \hfill Baltimore 2018
    \item[$\bullet$] Optimal Optical Coronagraphs workshop (SOC \& LOC) \hfill  Leiden, 2017
    \item[$\bullet$] High Contrast Imaging from Space (SOC) \hfill  Baltimore, 2016
\end{itemize}

\vspace{0.1cm}
\textbf{Autres investissements}
\vspace{-0.1cm}
\begin{itemize} \itemsep -2pt
    \item[$\bullet$] \textbf{Responsable de l'équipe ``Systèmes Exoplanètaires''} du LIRA \hfill Depuis 2025
    \item[$\bullet$] Participation au \textbf{Telescope Allocation Committee} d'Hubble \hfill 2024
    \item[$\bullet$] Comité d'experts du thème transverse (CET) exoplanètes de l'INSU  \hfill 2023 - 2024
    \item[$\bullet$] Comité Scientifique de l'action Spécifique Haute résolution Angulaire de l'INSU \hfill Depuis 2021
    \item[$\bullet$] \textbf{Relecteur} pour \textit{AJ}, \textit{A\&A}, \textit{MNRAS}, \textit{PASP} et \textit{JATIS}.
\end{itemize}



%%%%%%%%%%%%%%%%%%%%%%%%%%%%%%%%%%%%%%%%%%%%%%%%%%%%%%%%%%%%%%%%%%%%%%%%%%%%%%%%%%%%%%%%%%%%%%%%%%%%%%%%%%%%%%%%%%%%%%%%
%%%%%% SECTION TEACHING EXPERIENCE
%%%%%%%%%%%%%%%%%%%%%%%%%%%%%%%%%%%%%%%%%%%%%%%%%%%%%%%%%%%%%%%%%%%%%%%%%%%%%%%%%%%%%%%%%%%%%%%%%%%%%%%%%%%%%%%%%%%%%%%%

\vspace{-0.8cm}
\textcolor{RoyalBlue}{\section{\large ENCADREMENTS}
\vspace{-0.2cm}\hrule}
\vspace{0.4cm}
% \textbf{Manuela Castañeda Medina} (PhD, ONERA/DOTA): co-direction avec L. Mugnier, ONERA \hfill \textbf{Depuis 2026}\\
\textbf{Lukas Delaye} (PhD, LIRA): co-direction avec A. Potier \hfill \textbf{Depuis 2025}\\
\textbf{Vito Squicciarini} (Postdoc, LIRA): co-encadrement avec A.-M. Lagrange \hfill \textbf{2022-2025}\\
\textbf{Yann Gutierrez} (PhD, LIRA): co-direction avec L. Mugnier, ONERA \hfill \textbf{2022-2025}\\
\textbf{Iva Laginja} (Postdoc, LIRA): CNES post-doctoral Fellow \hfill \textbf{2022-2024}\\
\textbf{Sophia Stasevic} (PhD, LIRA) co-direction avec A.-M. Lagrange and J. Milli \hfill \textbf{2021-2025}\\
\textbf{Justin Hom} (PhD, ASU) co-encadrement avec J. Patience \hfill \textbf{2019-2023}\\
\textbf{Kevin Fogarty} (PhD, JHU) co-encadrement avec L. Pueyo \hfill \textbf{2017-2019}\\


\vspace{-0.5cm}
\textcolor{RoyalBlue}{\section{\large ENSEIGNEMENTS}
\vspace{-0.2cm}\hrule}
\vspace{0.4cm}
% \vspace{-0.2cm}
Cours de Master (Observatoire de Paris):
\begin{itemize} \itemsep -2pt
    \item[$\bullet$] Instrumentation for Astronomy 
    \item[$\bullet$] Detection of Exoplanets (collab. Anne-Marie Lagrange)
\end{itemize}


\end{document}